\section{Integer Square Root Algorithms}

\subsection{Algorithm Design Principles}

The algorithm design philosophy is straightforward:
provably correct algorithms which are efficient within the
Ethereum Virtual Machine
without any additional storage;
this includes using no lookup values for initialization.
It would be interesting to extend the results of~\cite{FormalVerIsqrt}
(which computes the integer square root of \texttt{uint64}
and uses table lookups)
to the \texttt{uint256} setting here.

\subsection{Specific Algorithms in Solidity}

\subsubsection{Algorithms Found Online}

The specific \textsc{Isqrt} algorithms we found online were
\Uniswap{} (Listing~\ref{list:uni-newton})~\cite{uniswap-v2},
\prb{} (Listing~\ref{list:prb-newton})~\cite{prb-math},
\OpenZeppelin{} (Listing~\ref{list:oz-newton})~\cite{open-zeppelin}, and 
\abdk{} (Listing~\ref{list:abdk-newton})~\cite{abdk-consulting}.
Two algorithms
(\prb{} (Listing~\ref{list:prb-newton}) and
\abdk{} (Listing~\ref{list:abdk-newton}))
appear to be exactly the same aside from different integer notation
and ``unchecked'' portions.
The remaining algorithm
\OpenZeppelin{} (Listing~\ref{list:oz-newton}) is equivalent
to \prb{} (Listing~\ref{list:prb-newton}) and
\abdk{} (Listing~\ref{list:abdk-newton})
outside of separating the function calls.

None of these algorithms reference a proof of correctness.
We note that \Uniswap{} (Listing~\ref{list:uni-newton})
is provably correct as it implements~\cite[Algorithm 1.7.1]{cohen1993}
(although it doesn't follow the comment
in~\cite[Chapter 1.7, Remarks]{cohen1993} that
``[w]hen actually implementing this algorithm,
the initialization step must be modified'').
Furthermore,
\OpenZeppelin{} (Listing~\ref{list:oz-newton})~\cite{open-zeppelin}
says that we ``need at most 7 iteration[s]''
while 
\abdk{} (Listing~\ref{list:abdk-newton})~\cite{abdk-consulting} states that
``[s]even iterations should be enough'';
\prb{} (Listing~\ref{list:prb-newton})~\cite{prb-math}
has similar language.

\lstset{
    showstringspaces=false,
    frame=single,
    numbers=none,
}
\lstinputlisting[
    label=list:uni-newton,
    float,
    language=Solidity,
    caption={Method for computing Integer Square Roots from \Uniswap{}~\cite{uniswap-v2}}]{code/uni-newton.sol}

\lstset{
    showstringspaces=false,
    frame=single,
}
\lstinputlisting[
    label=list:prb-newton,
    float,
    language=Solidity,
    caption={Method for computing Integer Square Roots from \prb{}~\cite{prb-math}}]{code/prb-newton.sol}

\lstset{
    showstringspaces=false,
    frame=single,
    numbers=none,
}
\lstinputlisting[
    label=list:oz-newton,
    float,
    language=Solidity,
    caption={Method for computing Integer Square Roots from \OpenZeppelin{}~\cite{open-zeppelin}}]{code/oz-newton.sol}

\lstset{
    showstringspaces=false,
    frame=single,
}
\lstinputlisting[
    label=list:abdk-newton,
    float,
    language=Solidity,
    caption={Method for computing Integer Square Roots from \abdk{}~\cite{abdk-consulting}}]{code/abdk-newton.sol}


\subsubsection{Provably Correct Algorithms}

Due to the requirement of seeking \emph{provably correct}
algorithms for computing integer square roots,
additional algorithms were written.

The first provably correct algorithm converts results from~\cite{PythonIsqrt}
into a Solidity algorithm;
we denote this as
\python{} (Listing~\ref{list:python-isqrt}).
The next algorithms come in two groups of 3:
one group is based on unrolled Newton iterations,
and the second group is based on Newton iterations using a \texttt{while} loop.
They are
\UnrolledOne{} (Listing~\ref{list:newton-unrolled-1}),
\UnrolledTwo{} (Listing~\ref{list:newton-unrolled-2}),
\UnrolledThree{} (Listing~\ref{list:newton-unrolled-3}),
\WhileOne{} (Listing~\ref{list:newton-while-1}),
\WhileTwo{} (Listing~\ref{list:newton-while-2}), and
\WhileThree{} (Listing~\ref{list:newton-while-3}).
We note that \UnrolledThree{} (Listing~\ref{list:newton-unrolled-3})
was originally presented in~\cite{EfficientIsqrt}.

Here, \Unrolled{} refers to the fact that the Newton iterations
are unrolled: a fixed number of iterations is performed;
\While{} refers to the fact that the Newton iterations are performed
within a \texttt{while} loop.
Label 1 refers to the fact that the initialization begins
at the largest power-of-2 less than or equal to integer square root value;
Label 2 refers to the fact that the initialization begins
at the smallest power-of-2 greater than the integer square root value;
Label 3 refers to the fact that the initialization begins
at the arithmetic mean of the power-of-2 above and below
the integer square root value.

We note that in \UnrolledOne{} and \UnrolledTwo{}
(Listings~\ref{list:newton-unrolled-1} and \ref{list:newton-unrolled-2}),
7 Newton iterations are performed;
in \UnrolledThree{} (Listings~\ref{list:newton-unrolled-3}),
only 6 Newton iterations are performed.
The results from~\cite{EfficientIsqrt}
show why the final check in \Unrolled{} ensures a valid result.
Also, in \WhileOne{} and \WhileThree{}
(Listings~\ref{list:newton-while-1} and \ref{list:newton-while-3}),
one Newton iteration is performed
before entering the \texttt{while} loop
(this is required for the stopping criterion);
this is not present in \WhileTwo{} (Listing~\ref{list:newton-while-2}).
From~\cite[Algorithm 1.7.1]{cohen1993},
no additional check is needed before returning the result.

\lstset{
    showstringspaces=false,
    frame=single
}
\lstinputlisting[
    label=list:python-isqrt,
    float,
    language=Solidity,
    caption={Provably correct method for computing Integer Square Roots based on~\cite{PythonIsqrt}; this algorithm was previously presented in~\cite{EfficientIsqrt}.}]{code/python_isqrt.sol}

\lstset{
    showstringspaces=false,
    frame=single,
    numbers=none,
}
\lstinputlisting[
    label=list:newton-unrolled-1,
    float,
    language=Solidity,
    caption={Provably correct method for computing Integer Square Roots (\UnrolledOne{})}]{code/newton_unrolled_1.sol}

\lstset{
    showstringspaces=false,
    frame=single,
    numbers=none,
}
\lstinputlisting[
    label=list:newton-unrolled-2,
    float,
    language=Solidity,
    caption={Provably correct method for computing Integer Square Roots (\UnrolledTwo{})}]{code/newton_unrolled_2.sol}

\lstset{
    showstringspaces=false,
    frame=single,
    numbers=none,
}
\lstinputlisting[
    label=list:newton-unrolled-3,
    float,
    language=Solidity,
    caption={Provably correct method for computing Integer Square Roots (\UnrolledThree{})}]{code/newton_unrolled_3.sol}

\lstset{
    showstringspaces=false,
    frame=single,
    numbers=none,
}
\lstinputlisting[
    label=list:newton-while-1,
    float,
    language=Solidity,
    caption={Provably correct method for computing Integer Square Roots (\WhileOne{})}]{code/newton_while_1.sol}

\lstset{
    showstringspaces=false,
    frame=single,
    numbers=none,
}
\lstinputlisting[
    label=list:newton-while-2,
    float,
    language=Solidity,
    caption={Provably correct method for computing Integer Square Roots (\WhileTwo{})}]{code/newton_while_2.sol}

\lstset{
    showstringspaces=false,
    frame=single,
    numbers=none,
}
\lstinputlisting[
    label=list:newton-while-3,
    float,
    language=Solidity,
    caption={Provably correct method for computing Integer Square Roots (\WhileThree{})}]{code/newton_while_3.sol}


\subsection{General Algorithm Discussion}

When using Newton iterations, it is critical to use
a good initial approximation to ensure a low overall cost.
Most of the algorithms use the same initialization value:
the largest power-of-2 less than or equal to the integer square root.
These are
\prb{} (Listing~\ref{list:prb-newton}),
\OpenZeppelin{} (Listing~\ref{list:oz-newton}),
\abdk{} (Listing~\ref{list:abdk-newton}),
\UnrolledOne{} (Listing~\ref{list:newton-unrolled-1}), and
\WhileOne{} (Listing~\ref{list:newton-while-1}).
Other good initialization values are 
\UnrolledTwo{} (Listing~\ref{list:newton-unrolled-2}) and
\WhileTwo{} (Listing~\ref{list:newton-while-2}) 
which follow the initialization suggestions of
\cite[Chapter 1.7, Remarks (2)]{cohen1993} and
\cite[Algorithm 9.2.11]{PrimeNumbersACP2005})
as well as
\UnrolledThree{} (Listing~\ref{list:newton-unrolled-3}) and
\WhileThree{} (Listing~\ref{list:newton-while-3})
(the new initialization from \cite{EfficientIsqrt}).
\Uniswap{} (Listing~\ref{list:uni-newton}) has the worst
initialization value (the input value itself).

For a better understanding of approximate square roots,
see~\cite{PythonIsqrt};
the discussion in~\cite{EfficientIsqrt} on approximate square roots
is based on~\cite{PythonIsqrt}.
