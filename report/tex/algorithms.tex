\section{Integer Square Root Algorithms}

\subsection{Algorithm Design Principle}

The algorithm design philosophy is straightforward:
provably correct algorithms which are efficient within the
Ethereum Virtual Machine.

\subsection{Specific Algorithms in Solidity}

\subsubsection{Online Algorithms}

The specific \textsc{Isqrt} algorithms we found online were
\Uniswap{} (Listing~\ref{list:uni-newton})~\cite{uniswap-v2},
\prb{} (Listing~\ref{list:prb-newton})~\cite{prb-math},
\OpenZeppelin{} (Listing~\ref{list:oz-newton})~\cite{open-zeppelin}, and 
\abdk{} (Listing~\ref{list:abdk-newton})~\cite{abdk-consulting}.
Two algorithms (\prb{} and \abdk{})
appear to be exactly the same aside from different integer notation
and ``unchecked'' portions.
The \OpenZeppelin{} algorithm is mathematically equivalent
to \prb{} and \abdk{}.

None of these algorithms reference a proof of correctness.
We note that \Uniswap{}
is provably correct as it essentially
implements~\cite[Algorithm 1.7.1]{cohen1993};
compare Listing~\ref{list:uni-newton} and Algorithm~\ref{alg:cohen_isqrt}.
Unfortunately, \Uniswap{} does not follow the comment
in~\cite[Chapter 1.7, Remarks]{cohen1993} which states that
``[w]hen actually implementing this algorithm,
the initialization step must be modified''.
\OpenZeppelin{}~\cite{open-zeppelin}
says that we ``need at most 7 iteration[s]''
while  \abdk{}~\cite{abdk-consulting} states that
``[s]even iterations should be enough'';
\prb{}~\cite{prb-math} has similar language.

\lstset{
    showstringspaces=false,
    frame=single,
    numbers=none,
}
\lstinputlisting[
    label=list:uni-newton,
    float,
    language=Solidity,
    caption={Method for computing Integer Square Roots from \Uniswap{}~\cite{uniswap-v2}}]{code/uni-newton.sol}

\lstset{
    showstringspaces=false,
    frame=single,
}
\lstinputlisting[
    label=list:prb-newton,
    float,
    language=Solidity,
    caption={Method for computing Integer Square Roots from \prb{}~\cite{prb-math}}]{code/prb-newton.sol}

\lstset{
    showstringspaces=false,
    frame=single,
    numbers=none,
}
\lstinputlisting[
    label=list:oz-newton,
    float,
    language=Solidity,
    caption={Method for computing Integer Square Roots from \OpenZeppelin{}~\cite{open-zeppelin}}]{code/oz-newton.sol}

\lstset{
    showstringspaces=false,
    frame=single,
}
\lstinputlisting[
    label=list:abdk-newton,
    float,
    language=Solidity,
    caption={Method for computing Integer Square Roots from \abdk{}~\cite{abdk-consulting}}]{code/abdk-newton.sol}


\subsubsection{Provably Correct Algorithms}

Due to the requirement of \emph{provably correct}
integer square roots algorithms,
additional algorithms were written.
All proofs are relegated to Appendix~\ref{app:proofs}.

The first provably correct algorithm converts results from~\cite{PythonIsqrt}
into a Solidity algorithm.
We denote this as
\python{} (Listing~\ref{list:python-isqrt});
this algorithm was previously presented in~\cite{EfficientIsqrt}.

The next algorithms come in two groups of 3.
They are
\UnrolledOne{} (Listing~\ref{list:newton-unrolled-1}),
\UnrolledTwo{} (Listing~\ref{list:newton-unrolled-2}),
\UnrolledThree{} (Listing~\ref{list:newton-unrolled-3}),
\WhileOne{} (Listing~\ref{list:newton-while-1}),
\WhileTwo{} (Listing~\ref{list:newton-while-2}), and
\WhileThree{} (Listing~\ref{list:newton-while-3}).
A version of \UnrolledThree{} was previously presented
in~\cite{EfficientIsqrt}.
Here, \unrolled{} refers to the fact that a fixed number of Newton iterations
are performed;
\while{} refers to the fact that the Newton iterations are iterated
within a \texttt{while} loop.
Labels 1, 2, and 3 refer to the different initialization values:
the largest power-of-two less than or equal to integer square root value;
the smallest power-of-two greater than the integer square root value;
the arithmetic mean of the two previous values.

Additional algorithms include
\BitLength{} (Listing~\ref{list:newton-bitlength}),
\Linear{} (Listing~\ref{list:newton-linear}),
\HyperFour{} (Listing~\ref{list:newton-hyper-4}),
\LookupFour{} (Listing~\ref{list:newton-lookup-4}), and
\LookupEight{} (Listing~\ref{list:newton-lookup-8}).
\BitLength{} uses the full bitlength of the number.
\Linear{} uses a linear approximation based on the high bits
of the input value.
\HyperFour{} uses a hyperbolic approximation
based on the high bits of the input value;
we investigated such approximations
based on the discussion in~\cite{WikiSqrtHyper}.
\LookupFour{} and \LookupEight{} use lookup tables
to obtain 4 bits and 8 bits of accuracy for the initial approximation;
the idea to use a lookup table for initialization
came from~\cite{WikiSqrtBinary,FormalVerIsqrt},
although~\cite{FormalVerIsqrt} uses a different method for
computing square roots
(it computes square roots by first computing the \emph{inverse square root}).

\lstset{
    showstringspaces=false,
    frame=single
}
\lstinputlisting[
    label=list:python-isqrt,
    float,
    language=Solidity,
    caption={Provably correct method for computing Integer Square Roots based on~\cite{PythonIsqrt}; this algorithm was previously presented in~\cite{EfficientIsqrt}.}]{code/python_isqrt.sol}

\lstset{
    showstringspaces=false,
    frame=single,
    numbers=none,
}
\lstinputlisting[
    label=list:newton-unrolled-1,
    float,
    language=Solidity,
    caption={Provably correct method for computing Integer Square Roots (\UnrolledOne{})}]{code/newton_unrolled_1.sol}

\lstset{
    showstringspaces=false,
    frame=single,
    numbers=none,
}
\lstinputlisting[
    label=list:newton-unrolled-2,
    float,
    language=Solidity,
    caption={Provably correct method for computing Integer Square Roots (\UnrolledTwo{})}]{code/newton_unrolled_2.sol}

\lstset{
    showstringspaces=false,
    frame=single,
    numbers=none,
}
\lstinputlisting[
    label=list:newton-unrolled-3,
    float,
    language=Solidity,
    caption={Provably correct method for computing Integer Square Roots (\UnrolledThree{})}]{code/newton_unrolled_3.sol}

\lstset{
    showstringspaces=false,
    frame=single,
    numbers=none,
}
\lstinputlisting[
    label=list:newton-while-1,
    float,
    language=Solidity,
    caption={Provably correct method for computing Integer Square Roots (\WhileOne{})}]{code/newton_while_1.sol}

\lstset{
    showstringspaces=false,
    frame=single,
    numbers=none,
}
\lstinputlisting[
    label=list:newton-while-2,
    float,
    language=Solidity,
    caption={Provably correct method for computing Integer Square Roots (\WhileTwo{})}]{code/newton_while_2.sol}

\lstset{
    showstringspaces=false,
    frame=single,
    numbers=none,
}
\lstinputlisting[
    label=list:newton-while-3,
    float,
    language=Solidity,
    caption={Provably correct method for computing Integer Square Roots (\WhileThree{})}]{code/newton_while_3.sol}

\lstset{
    showstringspaces=false,
    frame=single,
    numbers=none,
}
\lstinputlisting[
    label=list:newton-bitlength,
    float,
    language=Solidity,
    caption={Provably correct method for computing Integer Square Roots (\BitLength{})}]{code/newton_bitlength.sol}

\lstset{
    showstringspaces=false,
    frame=single,
    numbers=none,
}
\lstinputlisting[
    label=list:newton-linear,
    float,
    language=Solidity,
    caption={Provably correct method for computing Integer Square Roots (\Linear{})}]{code/newton_linear.sol}

\lstset{
    showstringspaces=false,
    frame=single,
    numbers=none,
}
\lstinputlisting[
    label=list:newton-hyper-4,
    float,
    language=Solidity,
    caption={Provably correct method for computing Integer Square Roots (\HyperFour{}) based on hyperbolic approximation}]{code/newton_hyper_4.sol}

\lstset{
    showstringspaces=false,
    frame=single,
    numbers=none,
}
\lstinputlisting[
    label=list:newton-lookup-4,
    float,
    language=Solidity,
    caption={Provably-correct method for computing Integer Square Roots (\LookupFour{})}]{code/newton_lookup_4.sol}

\lstset{
    showstringspaces=false,
    frame=single,
    numbers=none,
}
\lstinputlisting[
    label=list:newton-lookup-8,
    float,
    language=Solidity,
    caption={Provably-correct method for computing Integer Square Roots (\LookupEight{})}]{code/newton_lookup_8.sol}


\subsection{General Algorithm Discussion}

When using Newton iterations,
a good initial approximation is critical to ensure a low overall cost.
The following algorithms use the largest power-of-two less than or equal
to the integer square root:
\prb{} (Listing~\ref{list:prb-newton}),
\OpenZeppelin{} (Listing~\ref{list:oz-newton}),
\abdk{} (Listing~\ref{list:abdk-newton}),
\UnrolledOne{} (Listing~\ref{list:newton-unrolled-1}), and
\WhileOne{} (Listing~\ref{list:newton-while-1}).
Another good initialization value is shared by
\UnrolledTwo{} (Listing~\ref{list:newton-unrolled-2}) and
\WhileTwo{} (Listing~\ref{list:newton-while-2}),
which follow the initialization suggestions of
\cite[Chapter 1.7, Remarks (2)]{cohen1993} and
\cite[Algorithm 9.2.11]{PrimeNumbersACP2005}).
\UnrolledThree{} (Listing~\ref{list:newton-unrolled-3}) and
\WhileThree{} (Listing~\ref{list:newton-while-3})
use the new initialization from \cite{EfficientIsqrt}.
\BitLength{} (Listing~\ref{list:newton-bitlength}),
\Linear{} (Listing~\ref{list:newton-linear}),
\HyperFour{} (Listing~\ref{list:newton-hyper-4}),
\LookupFour{} (Listing~\ref{list:newton-lookup-4}), and
\LookupEight{} (Listing~\ref{list:newton-lookup-8})
use initializations specific to each method.
\Uniswap{} (Listing~\ref{list:uni-newton}) has the worst
initialization value (the input value itself).

\python{} (Listing~\ref{list:python-isqrt})
is based on approximate square roots~\cite{PythonIsqrt}.
