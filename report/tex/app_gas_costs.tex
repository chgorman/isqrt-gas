\section{Gas Costs}
\label{app:gas_costs}

We look at the specific gas costs of some operations;
these will be useful in later discussions when making comparisons
between different methods.

\subsection{Initializations}
\label{app_gas_costs:initializations}

We look at initialization costs associated with the algorithms;
see Table~\ref{table:initialzation_gas_costs} for the results.

Some of the initializations are self-explanatory.
We note that \texttt{Newton1} is used by \UnrolledOne{} and \WhileOne{};
\texttt{Newton2} is used by \UnrolledTwo{} and \WhileTwo{};
\texttt{Newton3} is used by \WhileThree{};
and \texttt{Newton3-Var} is used by \UnrolledThree{}.
The test used the value $2^{256}-1$ to compute the gas cost.
We ignore return logic but include gas costs associated
with getting the input to and from the proper range;
this is used in \Linear{}, to give one example.

\texttt{Newton} and \texttt{Newton-Var} have similar gas values;
a notable difference is seen in 
\texttt{Newton3} and \texttt{Newton3-Var},
and this is why \UnrolledThree{} uses the \texttt{Newton3-Var} initialization.
\Linear{} has the lowest initialization out of the non-\texttt{Newton}s;
this (coupled with only needing 5 Newton iterations)
shows why this method had such good results in Table~\ref{table:gas_costs_3}.
This also suggests that more complex initializations will
not lead to overall improvements,
as \HyperFour{} gives the same accuracy and is only slightly more complicated
yet costs significantly more gas.
It is interesting to see that \LookupFour{} and \LookupEight{}
have such expensive initializations;
this suggests that no lookup-table-based method will be competitive.

\begin{table}[p]
\centering
\begin{tabular}{|c|S[table-format=3.0]|}
\hline
Method & Gas \\
\hline
\texttt{Newton1}     & 424 \\
\texttt{Newton2}     & 417 \\
\texttt{Newton3}     & 438 \\
\texttt{Newton1-Var} & 420 \\
\texttt{Newton2-Var} & 424 \\
\texttt{Newton3-Var} & 420 \\
\BitLength{}         & 468 \\
\Linear{}            & 454 \\
\HyperFour{}         & 500 \\
\LookupFour{}        & 577 \\
\LookupEight{}       & 628 \\
\hline
\end{tabular}
\caption[Initialization Gas Costs]{Here we give
    the gas cost for initialization methods;
    note that we ignored the initial check and did not include
    return logic.
    \texttt{Newton1} is used by \UnrolledOne{} and \WhileOne{};
    \texttt{Newton2} is used by \UnrolledTwo{} and \WhileTwo{};
    \texttt{Newton3} is used by \WhileThree{};
    and \texttt{Newton3-Var} is used by \UnrolledThree{}.
    The value $2^{256}-1$ was used.
    This clearly shows that some of the more complex initialization methods
    cost more gas.
    }
\label{table:initialzation_gas_costs}
\end{table}


\subsection{Iterations}
\label{app_gas_costs:iterations}

\begin{table}[p]
\centering
\begin{tabular}{|c|S[table-format=3.0]|}
\hline
Newton Iterations & Gas \\
\hline
0 & 336 \\
1 & 391 \\
2 & 439 \\
3 & 487 \\
4 & 535 \\
5 & 583 \\
6 & 631 \\
\hline
\end{tabular}
\caption[Newton Iteration Gas Costs]{Here we give
    the gas cost for computing each Newton iteration.
    The difference between $k$ and $k+1$ iterations is 48 gas
    (notwithstanding the difference between $0$ and $1$,
    which is $55$).
    The value $3\cdot2^{160}$ was used.
    }
\label{table:newton_iteration_gas_costs}
\end{table}

\begin{table}[p]
\centering
\begin{tabular}{|c|c|S[table-format=4.0]|}
\hline
Value & While Iterations & Gas \\
\hline
$2^{256} - 2^{128}$ & 1 &  612 \\
$2^{256} - 2^{224}$ & 2 &  702 \\
$2^{256} - 2^{240}$ & 3 &  792 \\
$2^{256} - 2^{248}$ & 4 &  882 \\
$2^{256} - 2^{252}$ & 5 &  972 \\
$2^{256} - 2^{254}$ & 6 & 1062 \\
\hline
\end{tabular}
\caption[While Iteration Gas Costs]{Here we give
    the gas cost for computing each \WhileTwo{},
    which allows us to determine the cost per \texttt{while} loop.
    The difference between $k$ and $k+1$ iterations is 90 gas.
    }
\label{table:while_iteration_gas_costs}
\end{table}

\begin{table}[p]
\centering
\begin{tabular}{|c|S[table-format=3.0]|}
\hline
Halley Iterations & Gas \\
\hline
0 & 336 \\
1 & 474 \\
2 & 609 \\
3 & 744 \\
4 & 879 \\
\hline
\end{tabular}
\caption[Halley Iteration Gas Costs]{Here we give
    the gas cost for computing each Halley iteration.
    As we can see, the difference between $k$ and $k+1$ iterations is 135 gas
    (notwithstanding the difference between $0$ and $1$,
    which is $138$).
    The value $3\cdot2^{160}$ was used.
    }
\label{table:halley_iteration_gas_costs}
\end{table}

\begin{table}[p]
\centering
\begin{tabular}{|c|c|S[table-format=5.0]|S[table-format=3.0]|S[table-format=3.0]|S[table-format=2.0]|S[table-format=3.0]|}
\hline
Value & Binary Iterations &
    \multicolumn{1}{c|}{Total Gas} & \multicolumn{1}{c|}{Total Diff} &
    \multicolumn{1}{c|}{Init Gas}  & \multicolumn{1}{c|}{Init Diff}  &
    \multicolumn{1}{c|}{Comp Diff} \\
\hline
$2^{ 3}$ & 2 &  595 &     & 293 &     &     \\
$2^{ 5}$ & 3 &  729 & 134 & 311 &  18 & 116 \\
$2^{ 7}$ & 4 &  851 & 122 & 317 &   6 & 116 \\
$2^{ 9}$ & 5 &  960 & 109 & 311 &  -6 & 115 \\
$2^{13}$ & 6 & 1100 & 140 & 335 &  24 & 116 \\
$2^{15}$ & 7 & 1221 & 121 & 341 &   6 & 115 \\
\hline
\end{tabular}
\caption[Binary Search Iteration Gas Costs]{Here we give
    the gas cost for computing each iteration within Binary Search.
    Computing the difference in iterations for binary search
    is more involved because we must consider the difference in
    initialization cost.
    \textbf{Total Diff} gives the difference between the total gas
    between $k-1$ and $k$ binary iterations;
    \textbf{Init Diff} gives the difference in initialization cost;
    \textbf{Comp Diff} gives the computed difference:
    $\textbf{Comp Diff} = \textbf{Total Diff} - \textbf{Init Diff}$.
    These results give an approximate iteration cost of 115 gas.
    }
\label{table:binary_iteration_gas_costs}
\end{table}

\begin{table}[p]
\centering
\begin{tabular}{|c|c|S[table-format=4.0]|}
\hline
Value & Interp Iterations & Gas \\
\hline
$2^{256} - 2^{128}$ & 1 &  871 \\
$2^{256} - 2^{180}$ & 2 & 1171 \\
$2^{256} - 2^{196}$ & 3 & 1471 \\
$2^{256} - 2^{224}$ & 4 & 1771 \\
$2^{256} - 2^{230}$ & 5 & 2071 \\
$2^{256} - 2^{236}$ & 6 & 2371 \\
\hline
\end{tabular}
\caption[Interpolation Search Iteration Gas Costs]{Here we give
    the gas cost for computing each iteration within Interpolation Search,
    which allows us to determine the cost per \texttt{while} loop.
    The difference between $k$ and $k+1$ iterations is 300 gas.
    }
\label{table:interpolation_iteration_gas_costs}
\end{table}



\subsubsection{Newton Iteration}
\label{app_gas_costs:newton}

We look at the data in Table~\ref{table:newton_iteration_gas_costs},
which lists the gas costs associated with varying
the number of Newton iterations;
the \texttt{Newton3} initialization is used,
although there is no return logic.
Each Newton iteration costs 48 gas,
although the difference between $0$ and $1$ Newton iteration is $55$ gas.
These tests used the value $3\cdot2^{160}$
(to enable a direct comparison with the Halley iteration results
in Appendix~\ref{app_gas_costs:halley}).

\subsubsection{While Iteration}
\label{app_gas_costs:while}

It is not as straightforward to determine the cost per \texttt{while} loop
within \WhileOne{}, \WhileTwo{}, and \WhileThree{},
as the total number of iterations must be known
(by running the analogous computation).
To do so, we choose values of the same bitlength and determined
the number of required iterations manually.
The results may be found in Table~\ref{table:while_iteration_gas_costs};
as we can see,
each \texttt{while} loop costs 90 gas,
which makes each iteration almost twice as expensive
as the unrolled version.

From this we see that while it is clear that \emph{certain}
integer square root values may be computed more efficiently
using \texttt{while} loops,
in general this results in a more expensive operation.

\subsubsection{Halley Iteration}
\label{app_gas_costs:halley}

We look at the data in Table~\ref{table:halley_iteration_gas_costs},
which lists the gas costs associated with varying
the number of Halley iterations;
the \texttt{Newton3} initialization is used,
although there is no return logic.
Each Halley iteration costs 135 gas,
although the difference between $0$ and $1$ Halley iteration is $138$ gas.
These tests used the value $3\cdot2^{160}$
(so there is no overflow during the computation).

Thus, even though fewer Halley iterations are required
to get to the same level of accuracy,
each Halley iteration is almost three times more expensive
than a Newton iteration.

\subsubsection{Binary Search Iteration}
\label{app_gas_costs:binary_search}

We look at the data in Table~\ref{table:binary_iteration_gas_costs},
which lists the gas costs associated with varying
the number of iterations within Binary search.
The computation is more complicated because getting a different
number of iterations requires numbers with different bit lengths,
so a valid comparison should also compute (or estimate) the initialization
cost (or rather, the initialization cost \emph{difference}).
In looking at the data, we estimate that each binary iteration
costs 115 gas.
This is more than twice as expensive as an unrolled Newton iteration
and still more expensive than a Newton iteration within a \texttt{while} loop.

\subsubsection{Interpolation Search Iteration}
\label{app_gas_costs:interpolation_search}

We look at the data in Table~\ref{table:interpolation_iteration_gas_costs},
which lists the gas costs associated with varying
the number of iterations within Interpolation search.
Each iteration costs 300 gas,
so this method is not competitive.
