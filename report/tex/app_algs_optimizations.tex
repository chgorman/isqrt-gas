\section{Additional Algorithms and Further Optimizations}
\label{app:algs_optimizations}

We now investigate methods for potential improvements.
We start by looking at improvements in initialization
before turning to general methods.
We include results from standard tests from Section~\ref{sec:comparison}
in Table~\ref{table:gas_costs_additional},
but none of the additional algorithms are competitive.

\begin{table}[pt]
\centering
\begin{tabular}{|c|
    |S[table-format=3.0]|S[table-format=5.0]
    |S[table-format=5.0]|S[table-format=5.0]|}
\hline
& \python{} & \multicolumn{1}{c|}{Binary Searh}
& \multicolumn{1}{c|}{Interpolation Searh} & \multicolumn{1}{c|}{Hardware} \\
\hline
Max    & 957 & 15342 & 22771 & 15640 \\
Mean   & 881 &  7846 &  6655 & 13031 \\
Median & 882 &  7759 &  5636 & 12994 \\
Std    &  43 &  4283 &  5274 &   653 \\
\hline
\end{tabular}
\caption[Gas Costs Statistics Additional]{Here are statistics
    related to the gas cost data.
    All of these algorithms are provably correct,
    but only \python{} is close to being competitive.
    These results are for the tests in Section~\ref{sec:comparison}.
    }
\label{table:gas_costs_additional}
\end{table}



\subsection{Additional Initializations}
\label{app:algs_opt_init}

\emph{Ceteris paribus},
a more accurate initialization will require fewer Newton iterations
than a less accurate initialization,
so we begin by discussing initialization improvements.

\subsubsection{Polynomial Approximations}
\label{app:algs_opt_polynomial}

The results of \Linear{} in Table~\ref{table:gas_costs_3}
shows that a linear (affine) approximation provides
a good initial approximation
by providing essentially 4 correct bits.
It is possible to extend the approximation
to higher degree polynomials;
however, the results in Appendix~\ref{app_gas_costs:initializations}
suggest the gas required to get a more accurate initialization
will cost more than the potential savings in reduced Newton iterations
(48 gas per iteration as noted in Appendix~\ref{app_gas_costs:newton}).

Specifically, \UnrolledThree{}, with an initial accuracy of 2 bits,
requires 6 Newton iterations;
the results in Tables~\ref{table:gas_costs_2} and \ref{table:gas_costs_3}
show that \Linear{}, with an initial accuracy of 4 bits,
requires 5 Newton iterations but ends up being about 10 gas more expensive.
In Table~\ref{table:initialzation_gas_costs},
we see the initialization of \HyperFour{} costs 46 more gas than \Linear{}.
To reduce the number of Newton iterations to 4
would require than the initialization would be able to use
more gas than \Linear{} but less than \HyperFour{};
the author is not convinced this is possible.

\subsubsection{Rational Approximations}
\label{app:algs_opt_rational}

The results \HyperFour{} in Table~\ref{table:gas_costs_3}
shows that hyperbolas provide a good initial approximation;
however, \HyperFour{} is not more efficient than \UnrolledThree{}
or \Linear{}.
The discussion in Appendix~\ref{app:algs_opt_polynomial} is worth reiterating:
Newton iterations are cheap and good approximations are expensive.
It seems unlikely that a rational function which gives 8 bits of accuracy
will result in a lower overall gas cost.

\subsubsection{Lookup Tables}
\label{app:algs_opt_lookup}

The results in Table~\ref{table:gas_costs_3}
show \LookupFour{} and \LookupEight{} produce decent results;
however, the gas costs are essentially the same even though \LookupEight{}
uses one fewer Newton iteration than \LookupFour{}.
This leads us to expect that higher precision
lookup tables (which would be ``\texttt{Lookup16}'')
will not result in lower gas costs
(not withstanding the cost of \emph{storing} all the approximations,
which would require a \emph{much larger} deployment cost).
Furthermore, in Table~\ref{table:initialzation_gas_costs}
we see \LookupFour{} and \LookupEight{} have
the most expensive initialization costs.


\subsection{Integer Square Roots via Inverse Square Roots}

In~\cite{FormalVerIsqrt}, the method for computing integer square roots
is actually based around computing \emph{inverse square roots}.
We note, though, that the setting and constraints are different
as~\cite{FormalVerIsqrt} focuses on the inverse square root algorithm
in order to not have to perform division.
As noted in Table~\ref{table:evm_gas},
within the Ethereum Virtual Machine
the cost of integer multiplication and division is the same.
Thus, although it would be an interesting exercise
to implement the algorithm in Solidity,
the author does not expect it to produce an efficient algorithm
(that is, better than \UnrolledThree{}).


\subsection{Approximate Square Roots}

One method based on \emph{approximate square roots}
may be found in \cite{PythonIsqrt}.
The algorithm is designated as
\python{} (Listing~\ref{list:python-isqrt});
this algorithm was previously presented in~\cite{EfficientIsqrt}.
The algorithm is provably correct
but is not as efficient as the other Newton-based methods;
see the results in Table~\ref{table:gas_costs_additional}.

\lstset{
    showstringspaces=false,
    frame=single
}
\lstinputlisting[
    label=list:python-isqrt,
    float,
    language=Solidity,
    caption={Provably correct method for computing Integer Square Roots based on~\cite{PythonIsqrt}; this algorithm was previously presented in~\cite{EfficientIsqrt}.}]{code/python_isqrt.sol}



\subsection{Halley's Method and Higher-Order Approximations}
\label{app_additional_algs:halley}

We have primarily focused on variations of Newton's method.
The asymptotic analysis in Appendix~\ref{app:asymptotics}
showed that fewer Halley iterations are required
in comparison to Newton iterations.
Unfortunately, in Appendix~\ref{app_gas_costs:iterations}
we saw that Halley iterations are almost three times more expensive
than Newton iterations
(notwithstanding implementation issues around overflow).

For completeness, we include a version of Halley's method;
see Listing~\ref{list:halley-unrolled-3}.
This method is of limited value as the code will overflow
when the input is too large,
and we have not proven the return logic is valid.
The initialization is the same as \WhileThree{},
and this is required in order to only have to perform 4 Halley iterations;
using the initializations of \UnrolledOne{} or \UnrolledTwo{}
would require 5 Halley iterations.
See Appendix~\ref{app:error_bounds:halley}
for the proof that 4 iterations is sufficient.

We include results of running Halley's method on $3\cdot2^{160}$ in
Table~\ref{table:halleys_method_results};
this is why the author
does not expect higher-order approximations
to produce practical improvements.

\begin{table}[pt]
\centering
\begin{tabular}{|c|S[table-format=3.0]|}
\hline
Method & Gas \\
\hline
Halley's Method  & 952 \\
\UnrolledThree{} & 705 \\
\hline
\end{tabular}
\caption[Halley's Method Gas Costs]{Here we
    compare running Halley's method from
    Listing~\ref{list:halley-unrolled-3}
    with \UnrolledThree{} (Listing~\ref{list:newton-unrolled-3});
    the value $3\cdot2^{160}$ was used.
    Even if it were possible to remove one Halley iteration,
    the total gas would only decrease from 952 to 817
    (which is still more expensive).
    }
\label{table:halleys_method_results}
\end{table}

\lstset{
    showstringspaces=false,
    frame=single,
    numbers=none,
}
\lstinputlisting[
    label=list:halley-unrolled-3,
    float,
    language=Solidity,
    caption={General form for computing Integer Square Roots based on Halley's method; the return logic of this method has not been verified and the values will overflow for large inputs.}]{code/halley_unrolled_3.sol}



\subsection{Search Algorithms}

We turn our attention to search algorithms.
The idea of using search algorithms comes
from~\cite[Chapter 11-1 Integer Square Root, Binary Search]{HackersDelight}.
At the end of the analysis,
we will see that these methods are not competitive;
this can be seen in the results in Table~\ref{table:gas_costs_additional}.

The noncompetitive nature appears to come from the
inherent complexity of search algorithms:
in~\cite[Page 425, Exercise 6.2.1.22]{TAOCPv3second},
Knuth mentions that on average $\lg\lg N$ comparisons are required
for interpolation sort
and that ``\emph{all} search algorithms \dots must make asymptotically
$\lg\lg N$ comparisons, on the average'' (emphasis in original).
The required $\lg\lg N$ comparisons would appear to translate into
$\lg\lg N$ loop iterations,
and each loop iteration will presumably be more complex
than a Newton iteration within the EVM;
in Appendix~\ref{app_gas_costs:binary_search},
we see the per-cost of each loop within binary search is
approximately 115 gas
(more than twice the cost of Newton).
This would appear to rule out all generic search algorithms
(even if better search algorithms only required $\lg\lg N$ iterations).

The paper referenced for the required $\lg\lg N$ comparisons on average
is presumably~\cite{YaoYaoSearching},
which the author of the present work is unable to access.

\subsubsection{Binary Search}

The idea of using \emph{binary search} to compute
the integer square root comes from its inclusion
in~\cite{HackersDelight};
see Listing~\ref{list:binary_search_C}
for the \texttt{C} code from~\cite{HackersDelightArchive}
and Listing~\ref{list:binary_search}
for the Solidity code.

Looking at Listing~\ref{list:binary_search},
we see that the initial bounds may be sufficient
when working with the \texttt{unsigned} type (usually \texttt{uint32}).
The \texttt{uint256} types on the EVM encourage a more accurate bounds
in order to reduce the number of loop iterations;
thus, we use powers-of-2 for the left and right limits.

By bounding the square root and then choosing the midpoint,
the error is halved at every iteration of the loop.
This leads to requiring $\Theta(\lg(x))$ number of iterations;
in particular, at most $\frac{1}{2}\lg(x)$ iterations.
Unfortunately, this means that 128 loop iterations are required
for the largest integer square root values we will compute,
and this leads to the large associated gas costs.
Thus, this method is not competitive and
no optimization will make it competitive.

\lstset{
    showstringspaces=false,
    frame=single,
    numbers=none,
}
\lstinputlisting[
    label=list:binary_search_C,
    float,
    language=C,
    firstline=86,
    lastline=98,
    caption={Binary search method for computing Integer Square Roots in \texttt{C} from~\cite{HackersDelight}}]{code/isqrt.c}

\lstset{
    showstringspaces=false,
    frame=single
}
\lstinputlisting[
    label=list:binary_search,
    float,
    language=Solidity,
    caption={Provably correct method based on binary search}]{code/binary_search.sol}


\subsubsection{Interpolation Search}

The $\Theta(\lg(x))$ number of iterations required by binary search
significantly increases the gas cost.
\emph{Interpolation search}~\cite{gonnet1980algorithmic}
seeks to reduce the required number of iterations
by using a linear approximation to update where to search next.
It is possible to show the required iterations is $\Theta(\lg\lg(x))$,
but this \emph{assumes a uniform search space}
(which is not the case here).
The results in Table~\ref{table:interpolation_iteration_gas_costs}
show the per-iteration cost of 300 gas per loop makes not a viable choice.
The Solidity code may be found in Listing~\ref{list:interp_search}.

The general idea of interpolation search makes sense:
you can search an ordered array more quickly (faster than binary search)
by estimating how the values in the array change with a linear approximation.
The problem is that we are not searching a generic array
but looking to compute a square root,
and we can compute the exact derivative (so no approximation is needed).

An analysis of interpolation search may be found
in~\cite{gonnet1980algorithmic},
and Algorithm~\ref{alg:interp_search} is based on the algorithm
from~\cite[Page 41]{gonnet1980algorithmic}.

While there are more complicated search algorithms possible
(like the
\emph{interpolated binary search} algorithm~\cite{mohammed2021interpolated}
which seeks to combine the best of both binary and interpolation search),
the results here suggest that no search method
will be competitive against Newton's method.

\begin{algorithm}[t]
\caption{Generic Interpolation Search
    from~\cite[Page 41]{gonnet1980algorithmic} (modified)}
\label{alg:interp_search}
\begin{algorithmic}[1]
\Require \textsf{array} is a sorted array of values;
    $\vphi(\alpha, n)$ is the probe position to inspect.
\Procedure{InterpolationSearch}{$\alpha$}
    \State $\textsf{low} \algAssign{} 0$
    \State $\textsf{high} \algAssign{} n+1$
    \While{$\textsf{high} - \textsf{low} > 1$}
        \State $j \algAssign{}
            \vphi((\alpha-\textsf{array}[\textsf{low}])/
                (\textsf{array}[\textsf{high}]-
                \textsf{array}[\textsf{low}]), 
                \textsf{high}-\textsf{low}-1) + \textsf{low}$
        \If {$\alpha = \textsf{array}[j]$}
            \State \Return $j$
        \ElsIf {$\alpha < \textsf{array}[j]$}
            \State $\textsf{high} \algAssign{} j$
        \Else
            \State $\textsf{low} \algAssign{} j$
        \EndIf
    \EndWhile
    \State \Return $\textsf{fail}$
\EndProcedure
\end{algorithmic}
\end{algorithm}

\begin{algorithm}[t]
\caption{Generic Interpolation Search
    from~\cite[Page 41]{gonnet1980algorithmic} (modified)}
\label{alg:interp_search}
\begin{algorithmic}[1]
\Require \textsf{array} is a sorted array of values;
    $\vphi(\alpha, n)$ is the probe position to inspect.
\Procedure{InterpolationSearch}{$\alpha$}
    \State $\textsf{low} \algAssign{} 0$
    \State $\textsf{high} \algAssign{} n+1$
    \While{$\textsf{high} - \textsf{low} > 1$}
        \State $j \algAssign{}
            \vphi((\alpha-\textsf{array}[\textsf{low}])/
                (\textsf{array}[\textsf{high}]-
                \textsf{array}[\textsf{low}]), 
                \textsf{high}-\textsf{low}-1) + \textsf{low}$
        \If {$\alpha = \textsf{array}[j]$}
            \State \Return $j$
        \ElsIf {$\alpha < \textsf{array}[j]$}
            \State $\textsf{high} \algAssign{} j$
        \Else
            \State $\textsf{low} \algAssign{} j$
        \EndIf
    \EndWhile
    \State \Return $\textsf{fail}$
\EndProcedure
\end{algorithmic}
\end{algorithm}



\subsection{Hardware Algorithm}

We include an additional algorithm based on one
from~\cite[Page 285]{HackersDelight};
see Listing~\ref{list:hardware_C}
for the \texttt{C} code from~\cite{HackersDelightArchive}
and Listing~\ref{list:hardware} for the Solidity code.
This is a ``hardware'' algorithm,
and the only operations used are comparisons, bitwise-or, bit shifts,
and subtraction.
The 128 loop iterations, though, make this not competitive;
see the results in Table~\ref{table:gas_costs_additional}.

\lstset{
    showstringspaces=false,
    frame=single,
    numbers=none,
}
\lstinputlisting[
    label=list:hardware_C,
    float,
    language=C,
    firstline=103,
    lastline=118,
    caption={``Hardware'' method for computing Integer Square Roots in \texttt{C} from~\cite{HackersDelight}}]{code/isqrt.c}

\lstset{
    showstringspaces=false,
    frame=single
}
\lstinputlisting[
    label=list:hardware,
    float,
    language=Solidity,
    caption={Provably correct method based on hardware-efficient operations}]{code/hackers_hardware.sol}



\subsection{Potential Optimizations}

Based on the cost of Newton iterations (48 gas per unrolled iteration
and 90 gas within a \texttt{while} loop;
see Appendix~\ref{app_gas_costs:iterations}),
reducing the number of Newton iterations only reduces the overall
gas cost so much.

\paragraph{Additional Approximations}
As mentioned in Appendix~\ref{app:algs_opt_init},
it is expected that polynomial estimates, rational approximations,
and lookup tables will not lead to an overall reduction in gas.
It is unclear where additional approximation methods would come from.
The best algorithms presented here
(\UnrolledThree{} (Listing~\ref{list:newton-unrolled-3})
followed closely by \Linear{} (Listing~\ref{list:newton-linear}))
produce 2 or 4 bits of initial precision.

\paragraph{Additional Algorithms}
We implemented algorithms based on Newton's method
and its generalizations, search algorithms,
and approximate square roots.
It is unclear to the author what other methods could lead
to a competitive algorithm.
From the algorithms investigated here,
the method would likely need to have quadratic convergence.

\paragraph{Assembly}
One way to reduce gas is to write all algorithms in assembly.
This is not completely separate from algorithm development,
as it may be the case that optimized (assembly) versions
of algorithms have lower gas costs which result in a different ranking.
This requires a thorough knowledge of Solidity assembly
and the result should be audited to ensure the algorithm
is properly implemented.
We leave this task to the interested reader.
