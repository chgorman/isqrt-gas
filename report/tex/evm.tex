\section{Ethereum and the Ethereum Virtual Machine}

Ethereum is a blockchain designed to have the ability
to perform arbitrary computations~\cite{EthereumYellowpaper}.
The Ethereum Virtual Machine (EVM) executes all transactions on Ethereum.
While Turing-complete,
all operations cost a certain amount of \emph{gas}
and there is a limit to the amount of gas included within each block.
Thus, the efficiency of all operations on Ethereum
are measured by their gas usage.

The fundamental type is a \texttt{uint256} object:
a 256-bit unsigned integer.
Table~\ref{table:evm_gas} lists the costs of various
integer and bit operations within the EVM.
The fact that basic integer and bit operations are around 5 gas
while every transaction has a base cost of $21\cdot10^{3}$ gas
means that integer and bit operations are inexpensive.
The fact that clean (dirty) writes costs $20\cdot10^{3}$ ($5\cdot10^{3}$) gas
means that storing data on Ethereum is significantly more expensive;
thus, it is beneficial to minimize storage.
Another notable feature is that multiplication and division
have the same cost;
this fact is nontrivial,
and we note the following from a paper on computing integer square roots:
``The absence of division primitives is the main reason why we look
for an approximation of the inverse square root rather than the
square root itself''~\cite[Section 2]{FormalVerIsqrt}.

\begin{table}[pt]
\centering
\begin{tabular}{|c|c|}
\hline
EVM Ops & Gas \\
\hline
    $+,-$ & 3 \\
    $\times,\div$ & 5 \\
    \textsf{and}, \textsf{or}, \textsf{xor}, \textsf{not} & 3 \\
    $<,>,=$ & 3 \\
    $\gg,\ll$ & 3 \\
\hline
    Transaction & $21\cdot10^{3}$ \\
    \textsf{sstore} (clean) & $20\cdot10^{3}$ \\
    \textsf{sstore} (dirty) &  $5\cdot10^{3}$ \\
    Gas Target  & $15\cdot10^{6}$ \\
    Gas Limit   & $30\cdot10^{6}$ \\
\hline
\end{tabular}
\caption[EVM Gas Costs]{Here are some
    of the gas costs for certain integer and bit operations within the
    Ethereum Virtual Machine~\cite{BlockSizesGasLimits,EthereumYellowpaper}.
    ``Transaction'' is the base gas cost of each transaction.
    ``\textsf{sstore} (clean)'' refers to writing a nonzero word
    (32 bytes) to the blockchain;
    ``\textsf{sstore} (dirty)'' refers to overwriting a nonzero word
    with another nonzero word.
    ``Gas Target'' is the target amount of gas
    included within a single Ethereum block;
    ``Gas Limit'' is the maximum amount of gas
    an Ethereum block may contain.
    }
\label{table:evm_gas}
\end{table}

