\section{Comparison of Gas Costs}
\label{sec:comparison}

Because all operations are performed within the Ethereum Virtual Machine,
the primary cost metric is the overall gas cost.
Thus, the focus is on minimizing the total gas used.
Ideally, the best algorithm will have the lowest maximum, mean, and median
gas costs.
We note that the results here are different from those
in~\cite{EfficientIsqrt} for three reasons:
first, some of the algorithms are different
(and we chose the most recent version of those found online);
second, we ensure that all input values are unique
(certain values may have been double counted);
third, different input values were chosen.

\subsection{Data Point Selection}

Any gas cost comparison requires the selection of \texttt{uint256}
values for input.
The values were chosen in this way:

\begin{itemize}
\item all numbers of the form $2^{k}$, $2^{k}-1$, and $2^{k}+1$
    for nonnegative values of $k$;
\item $v-1$, $v$, and $v+1$ for $v = (2^{128}-1)^{2}$
    (these were chosen because of an edge case in the new algorithms);
\item random values chosen according to the
    loguniform distribution~\cite{ScipyLoguniform}
    on $\brackets{1, 2^{256}}$ when the random seed
    is initialized to $0$~\cite{NumpyRandomSeed}.
\end{itemize}

\noindent
The number of random samples were increased until
there were $2048$ unique values.
While it is clear that the particular distribution affects the results,
we believe choosing the input values in this way reduces
the risk of bias.
There were a total of $768$ deterministic values chosen
and a total of $1303$ random samples determined
the remaining 1280 values.

In addition to measuring the gas cost,
the result of each integer square root operation was
validated using
Python's integer square root~\cite{PythonIsqrt,PythonIsqrtLink}.
There was no instance in which an algorithm ever produced an incorrect result;
this is, of course, not a proof that 
\prb{}, \OpenZeppelin{}, and \abdk{} are correct,
but rather that there are no known values where these algorithms fail.

\subsection{Summary Statistics}

\begin{table}[p]
\centering
\begin{tabular}{|c|
    |S[table-format=5.0]|S[table-format=3.0]
    |S[table-format=4.0]|S[table-format=3.0]
    |S[table-format=3.0]|}
\hline
& \Uniswap{} & \prb{} & \OpenZeppelin{} &
    \abdk{} & \python{} \\
\hline
Max    & 34205 & 878 & 1021 & 881 & 959 \\
Mean   & 17734 & 795 &  950 & 803 & 883 \\
Median & 17639 & 798 &  949 & 803 & 884 \\
Std    &  9558 &  34 &   30 &  33 &  43 \\
\hline
\end{tabular}
\caption[Gas Costs Statistics 1]{Here are statistics
    related to the gas cost data.
    The \Uniswap{} and \python{} algorithms are provably correct.
    These results are for the tests in Section~\ref{sec:comparison}.
    }
\label{table:gas_costs_1}
\end{table}

\begin{table}[p]
\centering
\begin{tabular}{|c|
    |S[table-format=3.0]|S[table-format=3.0]
    |S[table-format=3.0]|S[table-format=4.0]
    |S[table-format=4.0]|S[table-format=4.0]|}
\hline
    & \UnrolledOne{} & \UnrolledTwo{} &
    \cellcolor{yellow!15} \UnrolledThree{} &
    \WhileOne{} & \WhileTwo{} & \WhileThree{} \\
\hline
Max    & 845 & 838 &                       817 & 1202 & 1154 & 1132 \\
Mean   & 769 & 768 & \cellcolor{yellow!15} 742 &  817 &  873 &  833 \\
Median & 773 & 773 & \cellcolor{yellow!15} 745 &  860 &  909 &  856 \\
Std    &  41 &  40 &                        40 &  175 &  155 &  135 \\
\hline
\end{tabular}
\caption[Gas Costs Statistics 2]{Here are statistics
    related to the gas cost data;
    all of these algorithms are provably correct.
    These results are for the tests in Section~\ref{sec:comparison}.
    }
\label{table:gas_costs_2}
\end{table}

\begin{table}[p]
\centering
\begin{tabular}{|c|
    |S[table-format=3.0]|S[table-format=3.0]
    |S[table-format=3.0]|S[table-format=3.0]
    |S[table-format=3.0]|}
\hline
  & \BitLength{} & \cellcolor{yellow!15} \Linear{} &
    \HyperFour{} & \LookupFour{} & \LookupEight{} \\
\hline
    Max    & 841 & \cellcolor{yellow!15} 812 & 845 & 922 & 925 \\
    Mean   & 768 &                       746 & 778 & 855 & 858 \\
    Median & 770 &                       751 & 784 & 861 & 864 \\
    Std    &  38 &                        37 &  38 &  41 &  41 \\
\hline
\end{tabular}
\caption[Gas Costs Statistics 3]{Here are statistics
    related to the gas cost data.
    All of these algorithms are provably correct.
    These results are for the tests in Section~\ref{sec:comparison}.
    }
\label{table:gas_costs_3}
\end{table}


A listing of the summary statistics describing the runs
may be found in Tables~\ref{table:gas_costs_1},
\ref{table:gas_costs_2}, and \ref{table:gas_costs_3}.
From here, we see that the algorithm with the smallest
maximum gas cost is \Linear{} while
the smallest mean and median gas costs is \UnrolledThree{}.
The standard deviations of \Linear{} and \UnrolledThree{} are reasonable.

\subsection{Detailed Analysis}

We now take a closer look at how the gas cost varies
with the argument.

For each value we tested,
we determined the minimal gas cost
and then counted the number of times each algorithm was minimal;
the results may be found in Table~\ref{table:minimal_gas_costs}.
Additionally, we show a histogram where certain algorithms
were minimal in Figure~\ref{fig:minimal_gas_hist}.
There does not appear to be a particular pattern
in the value distribution.
The main trend is that \UnrolledThree{}
consistently performs well across the range of arguments;
\Linear{} performs well, too.
The only region where this does not hold is for small arguments,
where other algorithms (\while{}) performed better;
this makes sense because those values require fewer Newton iterations.

Overall, \UnrolledThree{} has the lowest overall cost
for a plurality of points (and close to lowest gas cost),
so we declare this to be the algorithm of choice.
A more extensive analysis may be found in
Appendix~\ref{app:extended};
the results are similar.

\begin{table}[p]
\centering
\begin{tabular}{|c|S[table-format=4.0]|}
\hline
Total & 2048 \\
\hline
\Uniswap{}       &    2 \\
\python{}        &    5 \\
\UnrolledOne{}   &    5 \\
\UnrolledTwo{}   &    5 \\
\cellcolor{yellow!25} \UnrolledThree{} & \cellcolor{yellow!25} 1222 \\
\WhileOne{}      &  386 \\
\WhileTwo{}      &  156 \\
\WhileThree{}    &  297 \\
\hline
\end{tabular}
\caption[Minimal Gas Costs Statistics]{Here are number of times
    each method had minimal gas costs;
    methods not included were never minimal.
    These results are for the tests in Section~\ref{sec:comparison}.
    }
\label{table:minimal_gas_costs}
\end{table}

\begin{figure}[p]
\centering
    \begin{subfigure}[t]{0.45\textwidth}
    \includegraphics[width=\textwidth]{plots/minimal_hist_Unrolled3.pdf}
    \end{subfigure}
    \begin{subfigure}[t]{0.45\textwidth}
    \includegraphics[width=\textwidth]{plots/minimal_hist_While1.pdf}
    \end{subfigure}

    \begin{subfigure}[t]{0.45\textwidth}
    \includegraphics[width=\textwidth]{plots/minimal_hist_While2.pdf}
    \end{subfigure}
    \begin{subfigure}[t]{0.45\textwidth}
    \includegraphics[width=\textwidth]{plots/minimal_hist_While3.pdf}
    \end{subfigure}
    \caption{Here we plot a histogram showing where each algorithm is minimal
        and provides more detail to the results
        in Table~\ref{table:minimal_gas_costs}.
        Each bin counts the total number instances where the algorithm's
        gas cost was minimal.
        These results are for the tests in Section~\ref{sec:comparison}.
        }
    \label{fig:minimal_gas_hist}
\end{figure}

