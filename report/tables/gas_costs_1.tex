\begin{table}[p]
\centering
\begin{tabular}{|c|
    |S[table-format=6.0]|S[table-format=6.0]
    |S[table-format=6.0]|S[table-format=6.0]
    |S[table-format=6.0]|}
\hline
& \Uniswap{} & \prb{} & \OpenZeppelin{} &
    \abdk{} & \OpenZeppelinTwo{} \\
\hline
Max      &                        33931 &    874 &   1015 &    877 &                          823 \\
Mean     &                        17591 &    791 &    944 &    799 &                          749 \\
Median   &                        17497 &    794 &    943 &    799 &                          751 \\
Std      &                         9482 &     34 &     30 &     33 &                           35 \\
\hline
Abs Cost & \cellcolor{yellow!15} 226284 & 277197 & 280341 & 281321 & \cellcolor{yellow!15} 271153 \\
Rel Cost & \cellcolor{yellow!15}  44133 &  95046 &  98190 &  99170 & \cellcolor{yellow!15}  89002 \\
\hline
\end{tabular}
\caption[Gas Costs Statistics 1]{Here are statistics
    related to the gas cost data;
    we also include the absolute deployment gas cost and
    relative deployment gas cost
    (absolute gas cost less 182151 gas,
    the deployment cost of an empty smart contract).
    The \Uniswap{} and \OpenZeppelinTwo{} algorithms are provably correct.
    These results are for the tests in Section~\ref{sec:comparison}.
    }
\label{table:gas_costs_1}
\end{table}
